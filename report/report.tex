% https://www.springer.com/gp/computer-science/lncs/conference-proceedings-guidelines
\documentclass[runningheads]{llncs}
%
\usepackage[T1]{fontenc}
% T1 fonts will be used to generate the final print and online PDFs,
% so please use T1 fonts in your manuscript whenever possible.
% Other font encondings may result in incorrect characters.
%
\usepackage{graphicx}
\usepackage{float}
\usepackage{hyperref}
\usepackage{amsmath}
\usepackage{parselines}
% Used for displaying a sample figure. If possible, figure files should
% be included in EPS format.
%
% If you use the hyperref package, please uncomment the following two lines
% to display URLs in blue roman font according to Springer's eBook style:
%\usepackage{color}
%\renewcommand\UrlFont{\color{blue}\rmfamily}
%\urlstyle{rm}
%
\begin{document}
%
\title{A System Analysis to Dkron Scheduler}


%
%\titlerunning{Abbreviated paper title}
% If the paper title is too long for the running head, you can set
% an abbreviated paper title here
%
\author{Miguel Albuquerque, 1105828\inst{1}\orcidID{1105828}}
\institute{Instituto Superior Técnico, Av. Rovisco Pais 1, 1049-001 Lisboa, Portugal
\email{miguel.albuquerque@tecnico.ulisboa.pt}}
%
\maketitle              % typeset the header of the contribution
%
\begin{abstract}
WRITE ABSTRACT HERE
\end{abstract}

\par
\text{Repository: \href{https://github.com/miguel-msa/dkron-study}{github.com/miguel-msa/dkron-study}}
\par
\text{Commit hash:}\textbf{PUT COMMIT HASH HERE}

\keywords{Dkron \and Distributed \and Benchmark \and Scalability \and REVIEW THIS .}
%
%
%
% %%%
% %%%
\begin{figure}
\centering
\includegraphics[width=0.25\textwidth]{media/dkron-logo.png}
%\caption{dkron Logo .} \label{fig1}
\end{figure}
\section{Introduction}
%brief presentation of the selected system and the justification for its choice.
%The report should also specify the git commit identifier, and the url and hash of the video that should be considered for evaluation purposes. Failure to do so will result in a grade penalty.

On the surface, Dkron provides an interface that:
\begin{enumerate}
    \item Has flexible tag-based job definition
    \item Allows for Distributed execution, where these jobs can be ran in a distributed mode. % todo: review this!
    \item Has no Single Point of Failure.
    \item Easy to deploy with built-in replicated storage (BuntDB) relying on the Raft protocol.
    \item Provides a Web-GUI for administration.
\end{enumerate}
The above characteristics, and more, are further explored in \ref{system_description}

\break
When selecting the system to analyse, some of the factors driving the ultimate choice were:
\begin{enumerate}
    \item Interest in exploring \href{https://go.dev/}{Go} source code.
    \item Understand more about a system that is not a database
    \item IS THIS OK? REVIEW THIS
\end{enumerate}
After finding some systems fulfilling these requirements, Dkron became the system that raised a curious question:
\textit{What is the performance of a scheduler, seemingly, focused on availability? and reliability?}

As this is not explored by Dkron, I became interested in finding out the implementation, e.g. scheduling approach, and extrapolate,
joined with benchmarking, its behavior, i.e., its performance, and scaling capacity.

~\\
Ultimately, Dkron is a golang written scheduler with specific particularities for some key use-cases.
Its claims are not bold, neither on its performance, nor on how it actually does scheduling. This opens an
opportunity to not only explore something unclaimed, but also on how Dkron found the
optimal point, if exists, between performance and reliability.

%Some of the claimed use-cases are:
%- Email delivery
%- Payroll generation
%- Bookkeeping
%- Data consolidation for BI
%- Recurring invoicing
%- Data transfer

% TODO
% ! TODO: this text needs to change
% TODO: these claims are NOT CAREFULLY CONSIDERED! REVIEW THIS!!!
~\\ The analysis will consider these use-cases and apply a ...benchmark... to analyse the performance and infer Dkron's
performance on workloads similar, as in trying to represent, to such use-cases.
% TODO: these claims are NOT CAREFULLY CONSIDERED! REVIEW THIS!!!

\section{System Description}
\label{system_description}

PUT HERE ALL THE relevant PERKS\&QUIRKS of Dkron in detail

Always available: Using the power of the Raft protocol, Dkron is designed to be always available. If the cluster leader node fails, a follower will replace it, all without human intervention.

Flexible targets: Simple but powerful tag-based target node selection for jobs. Tag node count allows to run jobs in an arbitrary number of nodes in the same group or groups.



% todo: make this a paragraph instead ???
\subsection{No Single Point of Failure}
This no SPOF characteristic is very relevant for systems that depend on workload automation to function i.e. the scheduler dependee
system might be fault-tolerant, but the scheduler itself might not, making these "fault-tolerant" systems, indirectly, not as
so - Dkron fixes this problem by, as they claim, being the only existing scheduler with no SPOF. % ! REPETITIVE TEXT

% ? Acceptable Performance --> we'll find out on tests
Therefore, with acceptable performance, whilst providing Reliability? and Availability?, Dkron is an interesting solution
for use-cases that must guarantee fault-tolerance.

\section{Experimental Design}
\section{Results}
\section{Conclusion}

% ! %%%
% ! %%% EXAMPLES BELOW
% ! %%%

\begin{table}
\caption{Table captions should be placed above the
tables.}\label{tab1}
\begin{tabular}{|l|l|l|}
\hline
Heading level &  Example & Font size and style\\
\hline
Title (centered) &  {\Large\bfseries Lecture Notes} & 14 point, bold\\
1st-level heading &  {\large\bfseries 1 Introduction} & 12 point, bold\\
2nd-level heading & {\bfseries 2.1 Printing Area} & 10 point, bold\\
3rd-level heading & {\bfseries Run-in Heading in Bold.} Text follows & 10 point, bold\\
4th-level heading & {\itshape Lowest Level Heading.} Text follows & 10 point, italic\\
\hline
\end{tabular}
\end{table}


\noindent Displayed equations are centered and set on a separate
line.
\begin{equation}
x + y = z
\end{equation}
Please try to avoid rasterized images for line-art diagrams and
schemas. Whenever possible, use vector graphics instead (see
%Fig.~\ref{fig1}).

\begin{figure}
\includegraphics[width=\textwidth]{fig1.eps}
\caption{A figure caption is always placed below the illustration.
Please note that short captions are centered, while long ones are
justified by the macro package automatically.} \label{fig1}
\end{figure}

\begin{theorem}
This is a sample theorem. The run-in heading is set in bold, while
the following text appears in italics. Definitions, lemmas,
propositions, and corollaries are styled the same way.
\end{theorem}
%
% the environments 'definition', 'lemma', 'proposition', 'corollary',
% 'remark', and 'example' are defined in the LLNCS documentclass as well.
%
\begin{proof}
Proofs, examples, and remarks have the initial word in italics,
while the following text appears in normal font.
\end{proof}
For citations of references, we prefer the use of square brackets
and consecutive numbers. Citations using labels or the author/year
convention are also acceptable. The following bibliography provides
a sample reference list with entries for journal
articles~\cite{ref_article1}, an LNCS chapter~\cite{ref_lncs1}, a
book~\cite{ref_book1}, proceedings without editors~\cite{ref_proc1},
and a homepage~\cite{ref_url1}. Multiple citations are grouped
\cite{ref_article1,ref_lncs1,ref_book1},
\cite{ref_article1,ref_book1,ref_proc1,ref_url1}.

\begin{credits}
\subsubsection{\ackname} A bold run-in heading in small font size at the end of the paper is
used for general acknowledgments, for example: This study was funded
by X (grant number Y).

\subsubsection{\discintname}
It is now necessary to declare any competing interests or to specifically
state that the authors have no competing interests. Please place the
statement with a bold run-in heading in small font size beneath the
(optional) acknowledgments\footnote{If EquinOCS, our proceedings submission
system, is used, then the disclaimer can be provided directly in the system.},
for example: The authors have no competing interests to declare that are
relevant to the content of this article. Or: Author A has received research
grants from Company W. Author B has received a speaker honorarium from
Company X and owns stock in Company Y. Author C is a member of committee Z.
\end{credits}
%
% ---- Bibliography ----
%
% BibTeX users should specify bibliography style 'splncs04'.
% References will then be sorted and formatted in the correct style.
%
% \bibliographystyle{splncs04}
% \bibliography{mybibliography}
%
\begin{thebibliography}{8}
\bibitem{ref_article1}
Author, F.: Article title. Journal \textbf{2}(5), 99--110 (2016)

\bibitem{ref_lncs1}
Author, F., Author, S.: Title of a proceedings paper. In: Editor,
F., Editor, S. (eds.) CONFERENCE 2016, LNCS, vol. 9999, pp. 1--13.
Springer, Heidelberg (2016). \doi{10.10007/1234567890}

\bibitem{ref_book1}
Author, F., Author, S., Author, T.: Book title. 2nd edn. Publisher,
Location (1999)

\bibitem{ref_proc1}
Author, A.-B.: Contribution title. In: 9th International Proceedings
on Proceedings, pp. 1--2. Publisher, Location (2010)

\bibitem{ref_url1}
LNCS Homepage, \url{http://www.springer.com/lncs}, last accessed 2023/10/25
\end{thebibliography}
\end{document}
